\documentclass[12pt]{article}
\usepackage{booktabs}
\title{A Sample \LaTeX Document}
\author{Math 300}
\date{October 3, 2022}
\begin{document}
\maketitle
\section{Typing text}
Since \LaTeX is a markup language, any text we type appears on the page,
unless it contains one of the nine reserved characters of \LaTeX, listed below.
\begin{verbatim}
\ { } & _ ^ % ~ #
\end{verbatim}
If we want those characters to appear on the page, we must precede them by
a backslash, or, in the special case of the backslash itself, we must use the
command \verb|\verb|. In math mode, we can use the command \verb|\backslash|.

Note that there are three kinds of dash-objects. Hyphens are very short,
and are typed the way you would expect, using ”-”. Dashes, as in the range
1–2, are wider, and are typed using --. If you feel a need for a wide—dash,
you can use \verb|---|.
\section{Units}
Lengths in \LaTeX can be given in a number of Units.\\
\begin{tabular}{ll}
cm & Centimeters\\
em & The width of the letter M in the current font\\
ex & The height of the letter x in the current font\\
in & Inches\\
pc & Picas 1pc = 12pt)\\
pt & Points (1in = 72pt)\\
mm & Millimeters\\
\end{tabular}

\pagebreak
\section{Space}
The most direct way to make spaces is simply to use the \verb|\hspace| and
\verb|\vspace| commands, for horizontal and vertical space, respectively. Each
takes one argument: a distance specification for the size of the space. The
width or height of the space can be positive or negative. Note that \verb|\vspace|
can only be used in vertical mode; that is, when you are starting a new para-
graph or starting a float or doing something else that shifts text vertically.\\
It will not work in a line.

\LaTeX also has a number of predefined spaces. To produce a space with
fixed width, and which cannot be used as a line break, you may use the \verb|~|
character. This would typically be used to separate initials of an author,
or in other situations where we don’t want to have a single letter or initial
ending a line. More often, we don’t mind a line break, and want the space
to shrink or grow according to the justification requirements on the line. In
that case we make a standard space using \verb|\| ; backslash-space.

There are wider stretchy spaces available to us: \verb|\quad| and \verb|\qquad|. There
is also a “thin space”: \verb|\,|. The following words are separated by a thin space,
a standard space, a quad, and a qquad, respectively.
\begin{center}
space\,space space \quad space \qquad space
\end{center}

There are also predefined vertical spaces:\quad \verb|\smallskip| \quad \verb|\medskip|, and
\verb|\bigskip|, that behave as their names imply.

There are also some exceptionally stretchy spaces that we can use to
push text around. For example, the following line is set using the command \verb|text\hfill text|. The line below it was set using \verb|text\hfill text\hfill text.|\\
text\hfill text\\
text\hfill text\hfill text\\
You get the idea: \verb|\hfill| makes enough space to fill the line in question com-
pletely. If more than one \verb|\hfill| appears on a line, then the two negotiate
over how much space they each get.
\section{Lines and boxes}
There are a variety of ways to make lines and boxes in \LaTeX. The most
basic is to make a horizontal rule using \verb|\hrule|.
\hrule
\pagebreak
\verb|\hrule\| makes a new line, and fills it up with a horizontal line. If you
don’t want an entire line, you can use \verb|\hrulefill| as in \hrulefill . This command works like \verb|\hfill|, but instead of filling with space, it uses a
horizontal rule to fill the line.

To make a box around some content, you can use the \verb|\framebox| command. The framebox command puts a box around its argument, so that
\verb|\framebox{text}| looks like \framebox{text}. It takes optional width and position ar-
guments, so that \verb|\framebox[3in][l]{text}| appears as\\ \framebox[3in][l]{text}.
\section{Margins}
This section is mistitled. \LaTeX does not really do margins, so much as it
places text. It uses several variables in placing the text, which we can set.
For example, to make the left margin on all even-numbered pages 0.5 inches
wider than the default (which is 1 inch), we would define\\
\verb|\evensidemargin=0.5in|.\\
A list of the variables we can set and their default values follows.
\begin{verbatim}
\evensidemargin
\oddsidemargin
\topmargin
\textwidth
\textheight
\parskip
\baselineskip
\end{verbatim}
\section{Tables}
Sometimes we need to typeset tables. For example, consider Table 1. Any
resemblance of the numbers in Table 1 to those from any authentic poll is
purely coincidental.
\section{Homemade commands}
It was the best of ideas. It was the worst of ideas. LATEX was conceived as
a programming language. This is what makes it harder to process using a
\begin{center}
Table 1: Results from a poll that probably never happened.
\begin{tabular}{ccc}
\toprule 
\multicolumn{3}{c}{\textit{What is your favorite animal}}\\
\bottomrule\\
\end{tabular}
\end{center}
\end{document}