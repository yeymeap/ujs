\documentclass[a4paper,12pt]{article}
\usepackage[left=2cm, right=2cm, top=2cm, bottom=2cm]{geometry}
\usepackage{xcolor}
\begin{document}
\begin{center}
\begin{huge}
10. Síkgeometria
\end{huge}
\end{center}
\section*{I. Elméleti összefoglaló}
\subsection*{Szögek, nevezetes szögpárok}
\begin{itemize}
\item Egy adott pontból kiinduló két félegyenes a síkot két részre bontja. Egy-egy ilyen rész neve szög-
tartomány, vagy szög. A két félegyenest a szög szárainak, közös kezdőpontjukat a szög csúcsá-
nak nevezzük.
\end{itemize}
\begin{itemize}
\item Szögfajták
\begin{itemize}[label=$\circ$]
\item Nullszögnek nevezzük azt a szöget, amelynek szárai egybeesnek, és szögtartományát a két
egybeeső szára alkotja.
\end{itemize}
\end{itemize}
\end{document}