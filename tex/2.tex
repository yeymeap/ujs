\documentclass[12pt]{article}
\title{Matematikai mód használata}
\author{B. B. B.}
\usepackage[magyar]{babel}
\usepackage{amsmath}
\usepackage{amsfonts}
\usepackage{amssymb}
\begin{document}
\maketitle
\section{Matematikai környezet}
Legyen a egy változó\\
Legyen $a$ egy változó\\
Legyen $amp$ egy változó\\
Legyen $amp ha$ egy változó\\
\subsection{Alsó és felső index}
$a^{2}$\\
$a_{2}$
\subsection{Kiemelt matematikai mód}
\begin{center}
Pitagorasz tétele
\end{center} 
$$ a^{2}+b^{2}=c^{2} $$
\subsection{Törtek}

$\frac{2}{3}+1\frac{2}{3}$\\
$\dfrac{1+\frac{1}{n}}{2}$
\newpage
\subsection{Matematikai jelek}
$A \cap B$\\
$A \longleftrightarrow B$\\
$\aleph_{0}$\\
$\lbrace c[b(a)]\rbrace$\\
$\alpha$\\
$\pi$\\
\subsection{Feladat}
$$
\sum_{n=1}^{\infty}n=1+2+\ldots+n+\ldots
$$

$$
\prod^{5}_{n=1}n=n! = 1 \ast 2 \ast \ldots \ast 5 = 120
$$

$$
\lim_{n\longrightarrow \infty} \left(1+ \dfrac{1}{n}\right)^n = e
$$

$$
\forall \quad x \in \mathbb{R} \textrm{-re érvényes} \quad x^{2}>0
$$

\subsection{Számozott egyenletek}
\begin{equation}
\sum_{n=1}^{\infty}n=1+2+\ldots+n+\ldots
\end{equation} \label{eg1}
\begin{equation}
\prod^{5}_{n=1}n=n! = 1 \ast 2 \ast \ldots \ast 5 = 120
\end{equation}\label{eg2}
\begin{equation}
\lim_{n\longrightarrow \infty} \left(1+ \dfrac{1}{n}\right)^n = e
\end{equation}\label{eg3}
\begin{equation}
\forall \quad x \in \mathbb{R} \textrm{-re érvényes} \quad x^{2}>0
\end{equation}\label{eg4}
\begin{equation}
a^{2}+b^{2}=c^{2}
\end{equation}
A \aref{eg5} a Pitagorasz tétel.

\subsection{Mátrix}
$$
\left(
\begin{array}{cccc}
1 & 2 & 3 & 4\\
5 & a & \frac{1}{2} & 8\\
9 & 10 & 11 & 12 \\

\end{array}\right)
$$

\end{document}